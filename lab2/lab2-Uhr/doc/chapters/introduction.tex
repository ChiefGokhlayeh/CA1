% !TeX root = ../main.tex
\chapter{Introduction}

Purpose of this document is to describe and explain the inner workings of the digital clock application, created as part of lab exercise 4.1 in the Computer Architecture course at the University of Esslingen.

A description of operation, functional and non-functional requirements; as well as design requirements are described in the lab exercise sheet. This document will focus on \emph{how} things are implemented, not \emph{why}.

The manual is structured in chapters, each describing an individual module. A module description consists of the public interface, used to interact with other modules; as well as implementation details.

\section{Module Overview}\label{sec:module_overview}

Figure~\ref{fig:module_overview} depicts a logical component view of the entire application.

Component \emph{Main Loop} sits atop the structure and is responsible for sequencing the application flow. It itself is timed and driven by the \emph{Tick Generator} interface, provided by the \emph{Wrapper} subsystem. Tick Generator provides two different ticks, one based on a \SI{1}{\second} period called \emph{Long Tick} and a secondary \emph{Short Tick} based on a \SI{10}{\milli\second} period.

All user interaction is controlled through the \emph{User Interface} (abbr.\ UI) component. This includes:
\begin{itemize}
    \item Buttons
    \item LCD
    \item LEDs
\end{itemize}
It also controls the operating modes, such as SET-mode and NORMAL-mode.

The UI component requires both, Short and Long Tick to facilitate all its functions, while all other components work on Long Tick alone.

The overall timekeeping functionality is provided by the \emph{Clock} component. It consumes a Long Tick, that is used to increment the internal seconds, minutes and hours counters. The current time can then be read out through its \emph{Current Time} interface.

Temperature measurements are performed by a dedicated component called \emph{Thermometer}. It enables measurements of the internal ADC, does the necessary conversion and provides the value via \emph{Get Last Measurement}. Taking measurements is also driven by Long Tick, and is done at the start of each new Long Tick update cycle.

Lastly the \emph{Wrapper} subsystem provides miscellaneous interfaces such as driving the LCD and converting \SI{16}{\bit} integers to decimal formatted ASCII strings. These functions are implemented in HCS12 Assembly and do not follow standard C calling convention, which is why wrapping is needed to facilitate safe C/Assembly inter-operation.

\begin{figure}[ht]
    \centering
    \includegraphics[width=\textwidth]{generated/components.png}
    \caption{Module Overview}\label{fig:module_overview}
\end{figure}
